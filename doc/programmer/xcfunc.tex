\chapter{XCFunc Module}
\label{xcfunc}

This module preserves all of exchange-correlation 
functionals files. Basically they are all Fortran files,
since we try to be in accordance with the tradition
so that we can only the Fortran functions from the 
other place.

\section{Basic Functional Handling}

Firstly, there's a question that how can we pass in
the functional information into this module here?
For example, the B3LYP functional has rho and gamma
DFT variables, and the M06 functional has an additional
tau variable. How can we know the functional 
information?

There are two steps. Firstly, we need to generate
an integer array in the xcvar class according 
to the functional information. This function is 
called ``generateFuncInfor''. Then an array is 
returned, in this array; the unused DFT variables
would be marked as -1, and the used DFT variables
are marked with 1. For example, for B3LYP functional
the returning array would be like:
\begin{align}
 infor[RHO]   &= 1 \nonumber \\ 
 infor[GAMMA] &= 1 \nonumber \\ 
 infor[TAU]   &=-1 \nonumber \\ 
 infor[LAP]   &=-1 \nonumber \\ 
 infor[EXRHO] &=-1 
\end{align}
This array is in fixed length(maximum number of variable
types).

In the second step, if you want to write your functional file 
in this module, you need to pass this information array 
into the Fortran subroutine. Additionally, you have 
to define some integer array to hold the final
functional derivatives information(in the example below,
it's d1vars and d2vars array):
\begin{verbatim}
subroutine ABC(..., INFOR)
IMPLICIT NONE
#include "varlist.inc"
#include "fderiv1.inc"
#include "fderiv2.inc"
INTEGER INFOR(*)
INTEGER D1VARS(N_FUNC_DERIV_1)
INTEGER D2VARS(N_FUNC_DERIV_2)

! initilize the functional derivatives array
CALL init_func_deriv_1(VAR_INFOR, D1VARS)
CALL init_func_deriv_2(VAR_INFOR, D2VARS)

...
...
...

POS_RA_GAA = D2VARS[ID_RA_GAA]
D2F(i,POS_RA_GAA) = ....

...
...
...
\end{verbatim}
Additionally, it's worthy to note that it's possible to 
combine the different order functional derivatives together
in one subroutine, just as the example shown above; else
you can just calculation one functional derivatives; either 
way is OK.

In the include folder, we have declared the variable position
information in the fderiv1.inc, fderiv2.inc etc. They are 
actually generated by the python file of fderiv.py in the python
folder in this module. We note, that the orders of the 
variable position defined in these head files should be 
\emph{exactly} same with the orders generating in xcvar class(
see the constructor of xcvar). Therefore, if you have more 
variables need to add in, then you need to modify the 
fderiv.py and generate the head files again.

On the other hand, there's another place you need to revise. 
It's the head file varlist.inc which is used in initializing 
the functional derivative information array(see the file of 
initfuncderives.f).


\section{Advanced Discussion for Functional Handling}




