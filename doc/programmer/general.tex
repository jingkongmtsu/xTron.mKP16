%
%  set up on Feb. 2014
%  1  component list: we do not need to detailed introduce them
%     unless it's needed;
%  2  xcfunc class introduction;
%  3  parallel class;
%
\chapter{General Module}
%
%
%
The general folder provides fundamental component functions, classes and 
data for all of other modules. It forms the foundation for the whole 
program.

\section{Components List in General Folder}

The general folder contains the following components:
\begin{description}
 \item [libgen.h] This is the head file handles the fundamental data types
 as well as global constants such as math threshold, all kinds of physical
 constants.
 \item [scalablevec.h] This head file defines the fundamental vectors based 
 on the scalable allocator and STL vectors.
 \item [GlobalInfor] This class gathers the global setting information.
 \item [ParameterParsing] This class parses the user's setting parameters 
 for a given class/module from user input file.
 \item [Exep] This class is responsible for exception handling for the whole
 program.
 \item [TextRead] This class parses the string information in unit of line.
 \item [WordConvert] This class manipulates the convertion between fundamental
 data types, such as string to integer/floating point number etc.
 \item [FileReadWrite] This class is responsible for binary file hanlding. 
 For a given directory and file name, it will read/write files in binary form.
 \item [xcfunc] This class parse the functional information per user's request.
 \item [LocalMemScr] This class manipulates local memory for a given working
 thread so that to avoid frequent memory allocation.
\end{description}

\subsection{Large File Support}

There's an additional point worthy to note, which is related to large 
file support in the file IO\footnote{Please refer to 
\url{http://en.wikipedia.org/wiki/Large_file_support} for more information}.

In the FileReadWrite class, we use fread and fwrite function to do the file
reading/writing work. They are implemented in the standard C library, and 
for the data length it uses size\_t as data type. Hence for the 64 bit system,
the large file support is intrinsically solved by using the size\_t. However,
for the 32 bit machine the problem remains. Therefore, for a 32 bit system
large file support is not available in our software and the program will 
crash if it's handling large file (larger than 4GB) on a 32 bit system.

However, such deficiency is not a quite serious problem since currently most 
of the system have already migrated from 32 bit system to 64 bit system. For
this reason we make a note here to remind ourself the existing of such problem.

\section{XCFunc Class}
%%
%%

\subsection{How to Design The XCFunc Class}
%
%

XCFunc class is used to parse the general quantum chemistry method name.
Such method is not only limited on density functionals, but also including 
these post-HF methods etc. In general, for any quantum chemistry method;
it could be characterized by two features, whether it has exchange 
components and whether it has correlation components. For traditional 
HF method, it only has exchange, for MP2 or B3LYP methods etc., they 
both have exchange and correlation components(MP2's exchange is HF). 
The XCFunc class is designed based on this general idea.

Furthermore, the functional is constructed based on ``VARIABLES''. For 
normal DFT calculation, the possible variables could be electron density,
kinetic density etc. For HF method, the energy is a functional of molecular
orbital. Therefore, currently we support the following variables in XCFunc
class\footnote{The reader could refer to the \ref{xcvar} for more information
about DFT variables}:
\begin{itemize}
 \item  electron density ($\rho$);
 \item  gradient of electron density ($\nabla\rho$);
 \item  kinetic density ($\tau$);
 \item  Laplacian ($\nabla\cdotp\nabla\rho$);
 \item  exchange energy density ($e^{ex}$);
 \item  molecular orbital.
\end{itemize}

It's worthy to note, that such classification is not rigorous. For example, the
exchange energy density is also molecular orbital based functional. Therefore
the classification is designed for the sake of convenience only. In this
program, all of HF, post-HF methods(CI, CC, MP2 etc. ) and the multi-reference
methods(MRCI,MCSCF etc.) will be termed as ``ORBITAL FUNCTIONAL''. Therefore,
it the methods contains HF components (like B3LYP) or contain the post-HF
components etc. \footnote{like these double hybrid functional, see
\url{http://computationalchemistry.wikia.com/wiki/Double_hybrid}} then such
methods will be orbital functional.

\subsection{How to Implement XCFunc Class}
%
%
XCFunc gets functional information from two resources: one is the user input file,
the other is the xcfunc.conf file which is contained in the setting file 
folder(see the appendix discussing environmental variable). 

To define the quantum chemistry methods, there are two ways. One way is to 
define it in the xcfunc.conf, the other way is to define it in the user 
input file. The xcfunc.conf file contains all of pre-defined standard methods
such as HF, Becke88, LYP etc. xcfunc.conf is designed to hold these stable 
methods. On the other hand, user could create his own methods at his 
convenience in the user's input file.

On the other hand, XCFunc accepts two forms of inputs to form the constructor.
One way is to simply pass in the exchange and correlation functional name
to the XCFunc class (for example, ``HF  MP2'' will do HF exchange and 
MP2 correlation). The other way is to get the functional name from 
input file and parse it in the xcfunc.conf, or to get functional definition 
directly from input file. Details could be referred from the xcfunc.cpp
in the general folder.

\section{LocalMemScr Class}

The LocalMemScr class is to provide a simple wrapper to manipulate
memory for working threads. The manipulation is concentrating on two points:
firstly, set up a local memory pool for a working thread; secondly, return
a usable memory position for a required length which is used by the thread.
Therefore, this class is used within the threads.

The reason to set up this class is to attentionally avoid memory allocation
inside the heavy calculation module, e.g. the integral calculations. In the 
ERI calculation, simple survey clearly demonstrates frequent memory allocation
and deallocation occupies a significant amount of time consumption. The purpose
of this class is to reduce such time consumption for the heavy calculation
modules.

\begin{comment}
 
HistDataMan Class is moved to the math folder, since it handles all of 
matrix or vector<Double> type information. It does not belong here.

\section{HistDataMan Class}
%%
%%
HistDataMan represents ``Historical Data manipulation''. This class 
is used to manipulate large size data such as matrix, vectors etc.
The user of the class can choose either to store the data in the 
memory, or store the data on the disk for future recovery. This class
provides a generic interface so that user need not to know where the 
data stores.

The HistDataMan class is actually some compromise between the reality and the
ideal programming styles. Ideally the data is better ``localized'', the
HistDataMan breaks it by setting up data in one place and use it in the other
place. Therefore it cause a potential ``coupling'' between the two places of
the codes.  However; to repeatedly calculate large data could be obviously a
drawback for programs (see \ref{suggestion_Implementation_data} for more
information).  Therefore, this class is used to hold these data who is
expensive for repeat calculation, and would be used in several different
places. This class is trying to monitor the creation/use of the data, so that
to minimize the potential hazard introduced by creating/using data in different
places.

It's strongly recommended that HistDataMan should be use with a good reason,
else it's better to use another way to skip the use of HistDataMan class.

\subsection{Rules for HistDataMan Class}
%
%
For safely constructing/using the data. HistDataMan class has a rule that 
the data inside this class is constructed ``only once''. The class user
could recover the data in the following codes, but the rewritten action
for same section data is prohibited. By setting up this rule, HistDataMan
class try to avoid the situation that the same data could be changed 
in many different places.

On the other hand, HistDataMan class does not provide the function to
``recover'' the constructor. If the HistDataMan object is set up, then to
recover the data should be with the same object. Such restriction is based on
the point that the creation/using data inside HistDataMan class should be close
with each other. User of this class can pass the object to another function for
recovering the data, but it's not allowed to recover data in totally
independent place.

Finally, since the class also monitors the status of stored data, therefore
a prerequisite data order is necessary storing/retrieving data. In this 
class, data is stored by following their sequence of storing order. For 
the spin-polarized data, the alpha and beta part of data are stored in
sequence for each section.

\subsection{Basic Features of HistDataMan Class}
%
%
By following the rules, the data construction/recovery could be divided
into two different processes depending on whether the data is saved 
on disk or saved in memory.

If the data is saved on disk, then when class user firstly set up the
data, HistDataMan class will erase all of existing data and create new empty
data folder for saving data files in future. When data is asked to 
save into HistDataMan, each data will be given a unique file name 
(associated with its number) for future recognition. If the data is 
saved into the memory, the given data is pushed back to the end of 
the list. Finally, the data information is stored in the HistDataMan
object.

Furthermore, since the input data could be either vector form or matrix form;
therefore for matrix form data; HistDataMan saves the symmetrical matrix
only with lower-triangular part. 

On the other hand, HistDataMan does monitor the data status for each input
data. If the input data is matrix form, then it's row and column will be 
recorded and future recovery. For vector form of data, its total length
will be recorded, too.

All of the data will be stored into binary file if it's saving to disk. 
On the other hand, currently it's only the Double type of data could be 
saved inside this class.

\subsection{Basic Data of HistDataMan Class}
%
%
For the HistDataMan class, it needs to know the following  data to 
construct the basic class:
\begin{itemize}
 \item some container to hold the data if it's stored in memory;
 \item the location where the data stores on disk;
 \item data status which is used to monitor the data length etc.;
 \item whether the data is spin-polarized. 
\end{itemize}

\subsection{Storing Data on Disk}
%
%
HistDataMan class is using FileReadWrite class to manipulating requirement 
of reading/writing data on disk. Generally, HistDataMan class knows the 
data file name, and where it stores; therefore it directs the FileReadWrite
class to manipulates the file. 

The files are structured in the following way:
\begin{itemize}
 \item Each user input file has its own scratch directory who is 
 defined in GlobalInfor class (this is the root directory);
 \item Each geometry has its own working directory in the   
 subdirectory level(directory name is the geometry section number);
 \item Each type of data manipulated by HistDataMan class 
 has its own folder even it only has one data inside. The data is 
 named by the section number together with its spin-polarized 
 situation.
\end{itemize}

\end{comment}


