%
% set up on Feb. 2014
% * data structure:
%   *  fundamental data types;
%   *  composite data types: why we use STL
% * exception handling;
%   *  how to deal with exception in our code;
%   *  general structure for handling exception
% * design module in this program;
%   *  how to define a module;
%   *  structure of module
% * parallel design:
%   *  hybrid of threads and processes;
%   *  memory management;
% 
%
\chapter{Fundamental Components Design}
\label{fundamental_components_design}

In this chapter we will discuss the design of fundamental components for this 
program. Why it was chosen, and how it was generally designed and implemented.

%%%%%%%%%%%%%%%%%%%%%%%%%%%%%%%%%%%%%%%%%%%%%%%%%%%%%%%%%%%%%%%%%%%%%%%%%%%%%%%%%%%
\section{Design Modularized Program}
%
%
%
The biggest challenge in programming is how to deal with complexity.
The real world is complicated, how to maintain a good framework in the 
software so that to make the program adapt to variety of requirement for change,
perhaps is the biggest challenge in the programming.

One of effective method to reduce the complexity in programming is to
design the modularized program\footnote{
\url{http://en.wikipedia.org/wiki/Modular_programming}}. After 
decomposing the program into orthogonal modules, the complexity is 
hereby encapsulated into the given module. Therefore, this program is designed
to be a modularized program.

\subsection{How to Form A Module}
%
% 1 each module should be a group of function/classes that are interacted 
%   with each other
% 2 module means self-encapsulate
% 3 module may have infor class as a component
% 4 module could be a group of functions, or a group of classes
%
A module is a group of classes and/or functions that are conceptually linking 
with each other. In the source directory each folder represents an individual 
module. The discussion in this note is also organized in terms of modules.

In general, module should be conceptually cohesive. One of an ideal feature 
for module is that module itself is self-encapsulated, other modules
only talk with the given interface of the module; therefore the whole program
is decomposed into orthogonal components.   

One of practical feature for modules is the information passing. Module need to
know some information so as to carry out a specific job. This program handles
the information passing for modules in two different ways. One way is to parse
the information from user given input. In this way, each module has its own
section of data defined in the input file by the user, and the running module
parses the input requirements before performing the practical jobs. The other
way is receiving job specification from its top modules in the process of
constructing.  For both of the two methods, in general we uses the ``infor''
class in the module to handle the information passing. In other words, the
information processing work is done inside the infor class of the module,
and module read out information from the infor class to perform job. Such 
technique is used for sophisticated modules, such as gints, xcints, scf etc.

A common way to define a module is through it's physical or mathematical
meaning. For example, ``gints'' module consists of several component classes 
that are aiming to solve analytical integrals based on Gaussian primitive
functions. The ``shell'' module is composed by the classes and functions
that are used to define shell structure and parse related shell properties.

One the other hand, module could be composed by a series of functions or
classes which are independent with each other, however, these functions or
classes performs similar functionality in view of the whole program. For
example, the ``general'' module is consist of a group of independent classes,
they all serve as global utility components for other modules.

\subsection{Structure of Modules}
%
%  four layers:
%  1  general, math etc.
%  2  gints, xcints result etc.
%  3  SCF etc.
%  4  script as driver 
%
%
This program is generally consist of four layers of modules. The ground layer
is composed of global utility functions and classes. Second layer involves 
the heavy computation module ``solely'' based on the first layer. The third
layer is composed of the modules which implement based on the second layer
modules. Finally, the top layer is the driver scripts that manipulate the third 
layer modules so that to perform the given computation job.

The modules inside each layer is given as below (the top layer is trivial so they
are not listed here):
\begin{description}
\item [first layer] The first layer is composed by modules that defined by ourself,
and outside libraries that the whole implementation is based on.
\begin{description}
 \item [STL]     Standard Template Library. We use this library to form all of 
 composite data types used in this program;
 \item [TBB]     Threading Building Blocks. This library is used for multi-threading
 work throughout the whole program;
 \item [BOOST]   BOOST C++ libraries. This library complements the features missing 
 in the standard C/C++ libraries, such as file system handling, lexical cast etc.
 \item [BLAS and LAPACK] BLAS and LAPACK functions support for the program;
 \item [general] self defined module to provide global utility data, functions and classes;
 \item [math]    self defined module to provide mathematical support functions;
\end{description}
\item [second layer]:
\begin{description}
 \item [shell]   defining shell structure and shell properties;
 \item [geom]    defining geometrical structure;
 \item [gints]   solving analytical integrals in terms of Gaussian primitive functions; 
 \item [xcints]  solving numerical integrals for XC part in terms of Gaussian primitive 
 functions; 
 \item [xcfunc]  code for calculating functional and functional derivatives in numerical
 way;
\end{description}
\item [third layer]:
\begin{description}
 \item [vdw]  VDW interaction calculation based on second layer modules;
 \item [scf]  SCF calculation based on gints, xcints etc.
\end{description}
\end{description}

\subsection{Structure of User Input File}
%
%
%


%%%%%%%%%%%%%%%%%%%%%%%%%%%%%%%%%%%%%%%%%%%%%%%%%%%%%%%%%%%%%%%%%%%%%%%%%%%%%%%%%%%
\section{Data Structure}
%
%
%

\subsection{Fundamental Data Types}
%
%
%
Computation chemistry software usually uses the following fundamental data 
types:
\begin{itemize}
 \item machine dependent unsigned integer (size\_t(see stddef.h));
 \item signed integer (32 byte one and/or 64 byte one);
 \item floating point number (32 byte one, 64 byte one or even more);
 \item normal string and character (no wide type);
 \item boolean value
\end{itemize}
size\_t is used to represent the size of an object. In theory the maximum
value of size\_t is determined by system's condition. On a 32 bit system,
size\_t is 32 bit unsigned integer so it's maximum value is $2^{32}-1$.
On 64 bit system size\_t is 64 bit unsigned integer and it's maximum 
value is $2^{64}-1$. Therefore, it's an machine dependent data type.
We use size\_t to be seamlessly linking with STL. 

The most important issue related to fundamental data types is the software 
portability. In general, program should be able to transfer its fundamental 
data from one type to another with minor change. To achieve this purpose, 
we use the ``typedef'' to redefine the data type. For changing one data type 
to another, in general what it needs is just modifying the head file and 
recompilation.

\subsection{Composite Data Types}
\label{composite_data_types}
%
%
%
Composite data types such as vector, set, map etc. are also widely used 
in the program, too. We choose to use the standard template library (STL),
which is considered to be highly efficient and reliable. In STL, the composite
data types is referred as ``container'', and STL also provides variety of 
algorithms corresponding to the container, such as searching, sorting etc.

By using STL, there's no need to manually handle the memory allocation
in the program. STL uses a template class named as ``allocator'' to perform
the memory allocation, therefore memory management is self-contained
in the STL (for more discussion, see the \ref{mem_management}).

The STL string class is used in this program to perform string parsing jobs.
We try to avoid using raw char type of variables.The typedef is also used 
for composite data types, too for better potability. 

%%%%%%%%%%%%%%%%%%%%%%%%%%%%%%%%%%%%%%%%%%%%%%%%%%%%%%%%%%%%%%%%%%%%%%%%%%%%%%%%%%%
\section{Memory Management}
\label{mem_management}
%
% * data racing in terms of memory, STL allocator is safe?
% * concept of scalable allocator
% * our rule to use allocator  
%
\subsection{Use Allocator for Memory Management}
%
%
%
In this program we use STL to handle all kinds of composite data types(see
\ref{composite_data_types}). In consequence, the memory management is hereby
performed by the allocator corresponding to the composite data types. 
In contrast to the ``NEW'' and ``DELETE'' to handle the memory allocation
and deallocation, allocator uses more sophisticated technique to perform
memory management in more efficient way (such as memory pool etc.). Therefore,
the raw ``NEW'' and ``DELETE'' way to handle memory should be abandoned.      

\subsection{Data Races in Memory Management}
%
%
%
Memory for system is global resource. In the multi-threading environment,
when threads races for the global resource, sometimes the same pieces of 
memory may occasionally be assigned to different threads so that it forms 
``Data races''\footnote{the other name is race condition, see 
\url{http://en.wikipedia.org/wiki/Race_condition} for more information}.
In this case, if the threads read/write the memory together, then it will
bring unexpected results.

Quoting from the C++ standard\footnote{see 
\url{http://www.cplusplus.com/reference/memory/allocator/} for more information},
it's only C++11 gives satisfied description to the allocator in terms of 
data race:
\begin{quote}
Except for its destructor, no member of the standard default allocator class 
template shall introduce data races. Calls to member functions that allocate 
or deallocate storage shall occur in a single total order, and each such deallocation 
shall happen before the next allocation (if any) in this order.
\end{quote}
Since in this program we use allocator to perform memory allocation or deallocation(see
the section of composite data types), therefore the C++ compiler should be carefully
selected so that it's obey the C++11 standard. 

\subsection{Scalable Allocator}
%
%
%
The standard allocator is currently designed for single-thread purpose use. Problem of 
scalability arises when threads might have to compete for a single shared pool in a way 
that allows only one thread to allocate at a time. The scalable Allocator could be used
to allocate/deallocate memory resource to avoid scalability bottlenecks\footnote{
See the paper\cite{berger2000hoard,michael2004scalable,hudson2006mcrt} for more details}.
On the other hand, the scalable allocator could be also used in single thread environment,
where it performs similarly with the traditional allocator.

The use of scalable allocator could greatly improve the efficiency of multi-threading
program. If the memory allocation takes \%20 percent of time in multi-threading program,
then according to Amdahl's law, the maximum speedup brought by the multi-threading
will be less than 5 times. In this case, the use of scalable allocator will relieve 
this bottleneck.

Heretofore the STL does not provide its own scalable allocator, and C++ standard(the newest
C++ standard so far is C++11) does not say anything related to it neither. However, we believe
in the future the scalable allocator will become a necessary feature for all mature allocator.
That is also the reason why we stick to use it in our program.

\subsection{How to Use Allocator in This Program}

We use scalable allocator for all of STL containers which are based on either fundamental 
data type (for example, vector<double>, set<int> etc.) or composite data types(for example, 
vector<AtomShell>in MolShell class etc.). Such manipulation is based on the consideration
that scalable allocator will use its memory pool to handle the requirement of memory
allocation and deallocation. The only exception is string class in STL. We uses its default
form and do not change it. Therefore for handling string parsing etc. you may need to 
keep on eye on it if it's working in multi-threading environment.

Since in this program we use TBB to parallel the threading part, therefore the TBB scalable
allocator is used throughout the whole program.

%%%%%%%%%%%%%%%%%%%%%%%%%%%%%%%%%%%%%%%%%%%%%%%%%%%%%%%%%%%%%%%%%%%%%%%%%%%%%%%%%%%
\section{Parallel Design}
%
%
%

\subsection{Hybrid of Threads and Processes}
%
%
%
The computational chemistry program is in general considered to be ``data parallel'' 
style\footnote{see \ref{hybrid_para_ints} for more discussion}. The parallelism in
this program is generally designed to be a hybrid of threads and processes. In
the multi-processes level, the data is distributed to different nodes, then for each
node the computation is divided among threads. 

In the program, each module may have its own parallel design. For the detailed
parallel design information, please refer to the module itself.

\subsection{Thread Safety}
%
%
%
Thread safety is a criteria for code running in the context of multi-threading programs. 
A piece of code is thread-safe if it only manipulates shared data structures in a manner 
that guarantees safe execution by multiple threads at the same time. There are various 
strategies for making thread-safe data structures (for more information, please refer
to \url{http://en.wikipedia.org/wiki/Thread_safety}).


%%%%%%%%%%%%%%%%%%%%%%%%%%%%%%%%%%%%%%%%%%%%%%%%%%%%%%%%%%%%%%%%%%%%%%%%%%%%%%%%%%%
\section{Exception Handling}
\label{exception_handling}
%
%
%
\subsection{Introduction}
%
%
%
In the running time of a program, sometimes an anomalous or exceptional event may
arise. Such an event could be a disaster which causes the program to crash, 
or it could an error which turns the computation result to be incorrect, 
or it could be just irregularity and does not change anything ultimately.  

The handling of such exceptional event is called ``exception handling''. C++
uses a set of instructions (namely ``try'', ``catch'' and ``throw'') to establish
the handling of exception. In the try block, an exception is tested and thrown
out, then the program turns to trace back the stack to find the place to ``catch''
the exception so that it could be processed. Such mechanism enables the exception 
triggering and processing to be separated, which makes the complicated exception 
event handling applicable.

The exception handling in C++ is considered to be an powerful tool. However, 
on the other hand it's considered to be that the benefits of using exceptions 
outweigh the costs\footnote{See the google c++ style guide}.

Now let's consider a general anomalous event arise in computation chemistry 
program. In general, the QC program accepts the input from user and it only 
interacts with the hardware or other programs/libs etc. as required by the user input
in the running time. Therefore an anomalous event will lead to three kinds 
of results:
\begin{itemize}
 \item cause the program to crash or in an abnormal state;
 \item program could still normally terminate, but results are ruined;
 \item cause the program to diverge from its normal flow, and results are correct.
\end{itemize}

In the first two cases the anomalous event is actually referred as an ``error''
to the program. Practically such an error reflects that there's bugs inside the 
program which is needed to be fixed. Therefore, the reasonable handling of 
the first two types of ``exception'' is to terminate the program, print out detailed
debugging information and send it back for bug fix.

The third case is actually an ``warning'' message to the program. It indicates
that the program need to be revised, even though it survives for this run. Therefore
the reasonable handling for this case is to print out warning sign and detailed 
debugging information for future revision.

In summary, to handle the exception in a QC program involves two steps:
\begin{enumerate}
 \item print out detailed information for debugging;
 \item decide whether to print out warning message, or completely terminate the program.
\end{enumerate}
Hence the exception handling in QC program is direct and simple. For this reason,
we do not use the exception handling provided in the standard C++ language. Instead,
a simple way to handle the exception is suggested in the following content.

\subsection{Handle Exception}
%
%
%
Addition to the above general discussion, the exception has another general feature: 
it can be standardized and categorized. Therefore it's able to set up an exception 
list and group the exceptions into categories. Furthermore, each exception could be 
mapped with an unique code as an ID. The ID will reflects the intrinsic nature of the 
exceptions, with the ID program can decide whether to terminate the program or enable 
it to be proceeded.

Based on the exception ID the program use some simple classes to collects and handle 
the debugging information. There's a base class to hold fundamental information to 
all exceptions, namely:
\begin{itemize}
 \item class name or namespace name;
 \item function name;
 \item exception code;
\end{itemize}
With these information it's able to locate where the exception come from and what type
is the exception.

Moreover, sometimes more information is demanded for debugging therefore it's able to 
derive other kinds of classes based on the ``base class''. For example, the exceptions
related to file handling requires the information of the file name as well as file 
directory. For these cases, it's easy to generate derived class to contain more 
debugging information.
