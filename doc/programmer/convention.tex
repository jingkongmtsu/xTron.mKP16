%
% set up Feb. 2014
% this chapter is organized as:
% 1  ground principles for programming to guide us;
% 2  from the principles, we will make the code conventions for this program
%
\chapter{Coding Conventions}

%%%%%%%%%%%%%%%%%%%%%%%%%%%%%%%%%%%%%%%%%%%%%%%%%%%%%%%%%%%%%%%%%%%%%%%%%%%%%%%%%%%
\section{Suggested Principles of Good Programming}
\label{principle}

The principles of good programming are closely related to principles of good
design and engineering. Here is some general principles we try to follow in our
program\footnote{The whole content is actually copied from Christopher Diggins's
Blog, see the original content: 
\url{http://www.artima.com/weblogs/viewpost.jsp?thread=331531}}.

\textbf{DRY(Do not repeat yourself)} - This is probably the single most fundamental
tenet in programming is to avoid repetition. Many programming constructs exist
solely for that purpose (e.g. loops, functions, classes, and more). As soon as
you start repeating yourself (e.g. a long expression, a series of statements,
same concept) create a new abstraction.
\footnote{\url{http://en.wikipedia.org/wiki/Don\%27t_repeat_yourself}}

\textbf{Abstraction Principle} - Related to DRY is the abstraction principle ``Each
significant piece of functionality in a program should be implemented in just
one place in the source code.''
\footnote{\url{http://en.wikipedia.org/wiki/Abstraction_principle_(programming)}}

\textbf{KISS (Keep it simple, stupid!)} - Simplicity (and avoiding complexity) should
always be a key goal. Simple code takes less time to write, has fewer bugs, and
is easier to modify.
\footnote{\url{http://en.wikipedia.org/wiki/KISS_principle}}

\textbf{Avoid Creating a YAGNI (You aren’t going to need it)} - You should try not to
add functionality until you need it.
\footnote{\url{http://en.wikipedia.org/wiki/YAGNI}}

\textbf{Do the simplest thing that could possibly work} - A good question to ask one's
self when programming is ``What is the simplest thing that could possibly
work?'' This helps keep us on the path towards simplicity in the design.
\footnote{\url{http://c2.com/xp/DoTheSimplestThingThatCouldPossiblyWork.html}}

\textbf{Do not make me think} - This is actually the title of a book by Steve Krug on
web usability that is also relevant in programming. The point is that code
should be easily read and understood with a minimum of effort required. If code
requires too much thinking from an observer to understand, then it can probably
stand to be simplified. \footnote{\url{http://www.sensible.com/dmmt.html}}

\textbf{Open/Closed Principle} - Software entities (classes, modules, functions, etc.)
should be open for extension, but closed for modification. In other words,
don't write classes that people can modify, write classes that people can
extend. \footnote{\url{http://en.wikipedia.org/wiki/Open_Closed_Principle}}

\textbf{Write Code for the Maintainer} - Almost any code that is worth writing is worth
maintaining in the future, either by you or by someone else. The future you who
has to maintain code often remembers as much of the code, as a complete
stranger, so you might as well always write for someone else. A memorable way
to remember this is ``Always code as if the person who ends up maintaining your
code is a violent psychopath who knows where you live.''
\footnote{\url{http://c2.com/cgi/wiki?CodeForTheMaintainer}}

\textbf{Principle of least astonishment} - The principle of least astonishment is
usually referenced in regards to the user interface, but the same principle
applies to written code. Code should surprise the reader as little as possible.
The means following standard conventions, code should do what the comments and
name suggest, and potentially surprising side effects should be avoided as much
as possible.
\footnote{\url{http://en.wikipedia.org/wiki/Principle_of_least_astonishment}}

\textbf{Single Responsibility Principle} - A component of code (e.g. class or function)
should perform a single well defined task.
\footnote{\url{http://en.wikipedia.org/wiki/Single_responsibility_principle}}

\textbf{Minimize Coupling} - Any section of code (code block, function, class, etc)
should minimize the dependencies on other areas of code. This is achieved by
using shared variables as little as possible. ``Low coupling is often a sign of
a well-structured computer system and a good design, and when combined with
high cohesion, supports the general goals of high readability and
maintainability''.
\footnote{\url{http://en.wikipedia.org/wiki/Coupling_(computer_programming)}}

\textbf{Maximize Cohesion} - Code that has similar functionality should be found within
the same component.
\footnote{\url{http://en.wikipedia.org/wiki/Cohesion_(computer_science)}}

\textbf{Hide Implementation Details} - Hiding implementation details allows change to
the implementation of a code component while minimally affecting any other
modules that use that component.
\footnote{\url{http://en.wikipedia.org/wiki/Information_Hiding}}

\textbf{Law of Demeter} - Code components should only communicate with their direct
relations (e.g. classes that they inherit from, objects that they contain,
objects passed by argument, etc.)
\footnote{\url{http://en.wikipedia.org/wiki/Law_of_Demeter}}

\textbf{Avoid Premature Optimization} - Don't even think about optimization unless your
code is working, but slower than you want. Only then should you start thinking
about optimizing, and then only with the aid of empirical data. "We should
forget about small efficiencies, say about 97 percent of the time: premature
optimization is the root of all evil - Donald Knuth.
\footnote{\url{http://en.wikipedia.org/wiki/Program_optimization}}

\textbf{Code Reuse is Good} - Not very pithy, but as good a principle as any other.
Reusing code improves code reliability and decrease development time.
\footnote{\url{http://en.wikipedia.org/wiki/Code_reuse}}

\textbf{Separation of Concerns} - Different areas of functionality should be managed by
distinct and minimally overlapping modules of code.
\footnote{\url{http://en.wikipedia.org/wiki/Separation_of_concerns}}

\textbf{Embrace Change} - This is the subtitle of a book by Kent Beck, and is also
considered a tenet of extreme programming and the agile methodology in general.
Many other principles are based on the concept that you should expect and
welcome change. In fact very old software engineering principles like
minimizing coupling are related directly to the requirement of making code
easier to change. Whether or not you are an extreme programming practitioner,
this approach to writing code just makes sense.
\footnote{\url{http://www.amazon.com/gp/product/0321278658}}

\begin{comment}

%%%%%%%%%%%%%%%%%%%%%%%%%%%%%%%%%%%%%%%%%%%%%%%%%%%%%%%%%%%%%%%%%%%%%%%%%%%%%%%%%%%
\section{Coding Conventions}
\label{code_convention}

By following the above general principles, here we provide a set of conventions
to follow.

%%%%%%%%%%%%%%%%%%%%%%%%%%%%%%%%%%%%%%%%%%%%%%%%%%%%%%%%%%%%%%%
\subsection{Conventions for Designing}
\label{conventions_design}
%
%
%

\textbf{Design Portable Software}

This is the natural requirement for software engineering. 

\textbf{Design Modularized Program} 

The biggest challenge in programming is how to deal with complexity.
The real world is complicated, how to maintain a good framework in the 
software so that to make the program adapt to variety of changes requirement,
perhaps is the biggest challenge in the programming.

One of effective method to reduce the complexity in programming is to
design the modularized program\footnote{
\url{http://en.wikipedia.org/wiki/Modular_programming}}. After 
decomposing the program into orthogonal modules, the complexity is 
hereby encapsulated into the given module. Therefore, \textbf{We 
recommend that the program should be writing for module-forming 
purpose}.

\textbf{Design Program by Following }  

%%%%%%%%%%%%%%%%%%%%%%%%%%%%%%%%%%%%%%%%%%%%%%%%%%%%%%%%%%%%%%%
\subsection{Conventions for Implementation}
\label{conventions_implementation}
%
%
%

\textbf{Use Good Library as much as We Can} 

Following the ``Do not repeat yourself'' principle, it's better not to 
create wheels again by yourself, except that you really have a good 
reason in hand. Therefore, use the reliable, good quality library,
which stands the test of the time. Use these libraries as 
much as we can.

Here is a list of libraries we currently used in the program:
\begin{itemize}
 \item  Standard Template Library in C++;
 \item  Some parts of Boost Library:
 \begin{itemize}
  \item Filesystem;
  \item Lexical Cast;
  \item Special Functions in Math;
 \end{itemize}
\end{itemize}




%%%%%%%%%%%%%%%%%%%%%%%%%%%%%%%%%%%%%%%%%%%%%%%%%%%%%%%%%%%%%%%
\subsection{Conventions for Software Maintenance}
\label{conventions_maintenance}
%
%
%


\end{comment}


