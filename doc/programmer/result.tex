%
%
%
\chapter{Result Module}

The result module describes variety kind of calculation results for
quantum chemistry calculation. Currently, the following 
type of results are supported:
\begin{itemize}
  \item Matrix in AO dimension, which is derived from the normal matrix class;
  \item MO result, which is derived from normal matrix class;
  \item density matrix, which is derived from matrix AO class
\end{itemize}
All of these classes are constructed based on their physical definition.
If two results are physically different, then it implies that the data
may behave differently. Therefore, it's better to set up different classes
to describe them. It's worthy to note that the inheritance relationship
between the classes are also based on their physical definition.

\section{Spin Consideration}
%
%

One of important feature for calculation results is the spin states.  Usually
the spin states is know before the practical calculation applies (for example,
In SCF calculation we will know it's restricted/unrestricted calculation in
advance). Therefore, the problem here is that how can we express the spin 
information for the results?

Here one important consideration is that it may not be preferable to ``expose'' 
the spin information outside the result, which implies that the class 
user need to handle the spin states of the results by themselves. However,
some times the class user may need to deal with the spin state directly. Therefore,
we need to provide an way to satisfy both of the two contrary demands.

Generally, in most of the cases the result is processed for both of the spin 
states. In other cases when class user requires to handle the given spin
state result, it's necessary that such assigned result could be successfully
returned to the class user. Based on such observations, we set up some general
principle to handle the spin information for the result:
\begin{itemize}
  \item Two classes are set up for dealing with spin information. One class is
  used to describe the result data for given spin state (which is lower level
  class), and the other is used to hold the aggregation of the spin data (which
  is upper level class). Since the spin information is ``symmetrical'' between
  alpha and beta, therefore the alpha spin state and beta spin state should share
  the same functions (that means, for the lower level class it's not needed to
  distinguish whether it's alpha or beta in its member functions). Therefore,
  such arrangement should be satisfied;
  \item When the class user need to handle both of the spin states data, the 
  upper level class should be passed to him and this should be good enough. 
  \item When the class user need to handle a given spin state data, a reference
  of lower level class is returned from the upper level class to the class user. 
\end{itemize}

Such principle also leads to another question, that how can we decide that 
the given function should be implemented into the lower level class, or 
into the upper level class? I think the answer just lies in the principle itself.
If the function is only designed for handle data without considering its 
spin character, then it's should be in the upper level class; else it should 
be in the lower level class.

\section{MtrxAO and MtrxsAO Class}
%
%

\subsection{Shell Code and Symmetrical Matrix}
%
%

\subsection{Relationship with MO Class}
%
%

\section{MO and MOs Class}
%
%

MO class is used to store the molecular orbital (MO) result. Basically, MO
class just contains the MO data for a given spin state. Therefore, it's a 
square/rectangular matrix hence it's a derived class from Mtrx class. The 
row of MO matrix is always the number of normal basis sets\ref{}, and the 
column of MO matrix is the number of molecular orbitals.

Comparing with the MtrxAO class, who's shape is primarily determined by it's 
row shell and column shell; the MO matrix may encounter a situation that 
the number of molecular orbitals would be less than the number of basis
set number, which is because the linear dependency of the basis sets.
Such possibility is carefully considered in the forming of MO data. Therefore,
originally the MO would be initialized with square matrix shape, and it's column
would be automatically revised if linear dependency exists in basis set data.



