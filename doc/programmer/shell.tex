%
% April 2014
%
%
\chapter{Shell Module}
%
%
%
The shell module consist of the following components:
\begin{itemize}
 \item raw shell data parsing and reading (parseshell.cpp);
 \item formal shell data forming (shell.cpp);
 \item spherical and Cartesian type of basis set data transformation
 and basis set normalization factor scaling (c2ptrans.cpp);
 \item utility functions for shell data (shellprop.cpp and util folder)
\end{itemize}

\section{Raw Shell Data Parsing}

leave it to the future....

currently we did not finish it

\section{Shell Class}
%
%
%
Shell forms the foundation for quantum chemistry calculation. It's the atomic
component for building wave functions in quantum chemistry. This program
takes the following method to organize the shell data for an arbitrary molecular 
system\footnote{This idea is suggested by Prof. Jing Kong}:


In terms of 
programming point of view, shell is a collection of basis set data
either from user-defined file or from standard basis set data file. Therefore,
to produce the shell data we basically performs: 
\begin{itemize}
\item Read in shell information(usually it's basis set name) from the 
user file, and then we parse the name to get the corresponding basis set
files(if the shell data is provided by user, then the basis set data 
file is just the user input file). This is done in parseshell.cpp;
\item Read in the raw shell data from the basis set files(see RawShell etc. 
in the parseshell.h);
\item Convert the raw shell data into the final shell data in the MolShell class etc.
\end{itemize}

Now the next question is how we organize the shell data? Right now 
the shell data is defined in such hierarchy:
\begin{center}
shell $\Rightarrow$ atom shell $\Rightarrow$ molecule shell
\end{center}

Here the shell refers to the normal shell data.
Atom shell is a collection of shells where they all belong to the given atom. 
Molecular shell is a collection of all atom shells where these atoms forms
the given molecule. 

\subsection{Running Calculation for Huge System: Split Shell into Pieces}

For achieving running huge molecular system, generally we need to divide
the whole calculation job into batches, and run the job for different batch.

Based on this idea, we may need to split the whole molecule shell into many pieces.
Each molecular shell piece would be the beginning point for the following calculation job.
In our code, we have a ``split'' function in molecule shell class to realize 
such function. After split function, we will create a new molecule shell
which is only a piece of the original data. 

How to consider the grain size in the splitting process? We note that 
we consider the atom shell as the basic unit in splitting. That is 
to say, one or several atom shells are grouped together to form a 
piece, and one atom shell is the smallest scale of splitting process.

Here we have a question, that how to connect the piece with the original
data? In considering this, it's natural to record the original atom shell 
index in the piece molecule shell(that's the reason that we have a variable
called ``index'' in atom shell class). Therefore, for each atom shell in the 
piece molecule shell, we are able to know it's original position.

\section{Shell Pair Module}
 
Shell pair is the pair between given shells. According to the Gaussian Product
theorem, that any two Gaussian functions combined is leading to a new Gaussian
function on a new center. Therefore, any shells composed by the Gaussian functions
will pair into a new set of Gaussian functions. This is the theoretical background
for shell pair. 
  
In quantum chemistry, shell pair is used in analytical integrals calculation. For 
example, the kinetic integrals, nuclear-electron attraction integrals, electron-
electron repulsion integrals etc. In the practical calculation, the most important
feature for shell pair is the screening function. That is to say, the screening 
process divide the shell pairs into two categories: one is significant shell pair
and the other is negligible shell pair. The significant shell pair is that both
of the two shells are able to overlap together so that to contribute to the integral
calculations, and the negligible shell pair is that two shells are far away from each
other so that they can not form effective overlap within given error range. Since
the Gaussian functions decays exponentially, therefore it could be well expected 
that most of shell pairs in a big molecule are negligible. Thus the concept of 
screening is very important for integral calculations.
  
On the other hand, shell pair may provide another function which is called ``sorting''.
Sorting is to organized the same type of shell pairs together. The criteria for sorting
of shell pair is usually comes to the angular momentum and contraction degree. That
is to say, the shell pairs with same type of angular momentum and contractions will
put together. The benefit for sorting process is that we could create balanced load
for integral calculation. However, the disadvantage for sorting is that it breaks the 
locality of the data. More specifically, the data in the shell pair may not be adjacent.
For example, the current shell pair may be from atom 1 and atom 3; but the next shell 
pair could be from atom 100 and atom 200. 
  
Based on this considering, we may not sort the shell pairs and arrange the shell pairs 
in adjacent way. For example, the shell pairs could be stored according to the atom 
pairs: (atom 1, atom 2), (atom 1, atom 3), (atom 1, atom 4) etc. Currently we do not have
good comparison between the two ways, sorting or not sorting. However, it could be 
an interesting point in the further study.
 
\subsection{Designs for Shell Pair Module}
  
Based on the general discussion above, we devised the shell pairs into the following
classes:
\begin{itemize}
  \item significant shell pair;
  \item shell pairs;
  \item batch shell pair information;
\end{itemize}
  
The significant shell pairs is to perform screening test for the given pair of 
shells. If the test is passed, then we store all of data for the given shell pair
for further use.
 
In the final shell pairs, we will form a vector to hold all of significant shell pair
data. Here we use vector rather than the list is because that we need randomly accessing
the data. 
 
In forming the shell pair data, besides the screening job we also need to divide 
the shell pairs data into batches. The criteria for batch dividing could be 
shell pair sorting(as we have discussed above), or other ways. Currently we just
use sorting method. The sorting process is performed by the batch shell pair
information class.

\subsection{Further Consideration for Dealing with Huge System}

Shell pair concept is some natural derivation based on the shell. Therefore,
in designing the shell pair it's natural that we should not distinguish whether
the input shell is given in ``full'' or given in ``part''. Therefore,

\section{Shell Transform and Scale Module}
  
This class is designed to handle the Cartesian and pure transformation work plus the
basis set scale work.
 
What kind of situations we need to handle for transformation and scale work?
Generally, for pure and cart transformation/scale work, there are two situations in
terms of shells:
\begin{itemize}
  \item the row shell and column shell are same;
  \item the row shell and column shell are different.
\end{itemize}
  
On the other hand, considering the requirement of transformation, there could be
three situations:
\begin{itemize}
  \item transformation/scale work is only done on row shell;
  \item transformation/scale work is only done on column shell;
  \item transformation/scale work is done on both row shell and column shell.
\end{itemize}
Therefore, we need to consider that how to handle all of these situations.
 
\subsection{Design for The Class}

Currently we can find two general ways to handle the work mentioned above.

\begin{description}
 \item[shell pair loop method] For this method, the transformation work
 is loop over shell pairs data block of the given matrix. We note that 
 it's only the lower triangular part of matrix is transformed in this 
 method, so it's only applied to symmetrical matrix;
\item[row column method] This method perform the transformation work
from column to row. The column of the matrix is transformed first, then
do the work to the row part.
\end{description}

Comparing with first method The second method has the following features:
\begin{itemize}
 \item Pro:
  \begin{enumerate}
   \item   the calls to dgemm is less than the other choice;
   \item   necessary for the work if only row/column of the matrix is converted;
   \item   also necessary for the work if the matrix has different row shell and 
           column shell data;
  \end{enumerate}
 \item Con:
   \begin{enumerate}
    \item for symmetrical matrix, two times data is converted;
    \item the matrix dimension could be not quite fit for dgemm work. For example,
		    we may need to take care of $5\times 20000$ dimension of matrix. We do not 
          know whether dgemm is efficient in this case?
   \end{enumerate}
\end{itemize}

On the other hand, the shell pair method has the following features 
by comparing with row column method:
\begin{itemize}
 \item Pro:
  \begin{enumerate}
   \item   parallel friendly. Since the matrix corresponding to shell pair is so small so 
           that we can do it in the GPU threads. Also we could do it in the openmp threads;
   \item   only half size of data is transformed, no additional work is done;
  \end{enumerate}
 \item Con:
   \begin{enumerate}
    \item more dgemm calls;
	 \item use is limited;
 \end{enumerate}
\end{itemize}
    
\subsection{Additional Notes} 
 
For SP shell block, we do nothing for either transformation work or the scale work.
Therefore, it’s necessary to identify the SP shell block for each atom in the work 
so that to reduce to work for transformation/scale.

For the transforming and scale work, there’s very good point for the parallel
implementation. That is, the data after converted will never be mixed up. That 
is to say, when we divide the work into threads, each threads will read different 
matrix block and then write into different block. They never work on the same block! 
therefore, threads are parallel independent.

Currently the transformation work is performed with additional memory use, which 
is roughly the same size of the matrix waiting to be transformed. Therefore, such
transformation is not in-place transformation work. Currently we do not think this
could be a problem even for huge system.



\begin{comment}

%
% since we have abandon the corresponding code in shellprop.cpp, 
% we also comment out the words here
%
 
\section{Shell Property Module} 
 
In the shell module, sometimes we may have some small functions that could 
be used across the whole program. For example, to calculate the number
of basis set functions according to the shell type, get shell name by
given angular momentum etc. For these functions, it's not appropriate 
to put them into classes because sometime we want to use them without 
any object. Therefore, we set up this file to hold these type of functions.
 
On the other hand, there are some classes used as accessory for the main
shell classes. For example, the shell property class. Such classes are 
also listed here.
 
\subsection{Multi-Shell Compounds}
 
Based on the shell concept, we can form the multi-shell compounds; that is 
to say, some physical quantity that possesses more than one shells. For example,
the shell pair is formed by two shells, the electron repulsion integral is 
composed by four shells(so called shell quartet). Therefore, we face a question
that how to express the angular momentum for these multi-shell compounds?

In dealing with this situation, we have a group of functions which is called 
``codeL'' and ``decodeL''. What does it mean? These functions are used to generate
a code to represent the angular momentum status for shell pair, shell quartet
etc. In these case, the angular momentum are compressed together to form a 
code, we use the code to designate the status of angular momentum. On the other
hand, we could decode it to restore the original angular momentum for each
component in the multi-shell compounds. Such function is necessary in the integral
calculation, since the integral files are organized according to the angular 
momentum types. Based on the code of angular momentum, now we are able to connect
the given multi-shell integral with it's given integral file.
 
Additionally, it's worthy to note that for the multi-shell compounds, it possesses
an property that for the same side(bra or ket), the shells could exchange between
each other without changing the value of these multi-shell compounds. For example,
for the two body overlap integral between S shell and D shell, we can have 
(S|D) = (D|S). Therefore, we do not need to write the integral codes for dealing with 
both (S|D) and (D|S) situation; we only need one file to do the job.
 
This brings benefits to reduce the number of integral files. Based on this property,
now we do not necessarily have integral files which cover all of angular momentum
combination. As for the above case, we can only has integral file to calculate (D|S) 
integrals, and the (S|D) is safe to use this file, too.
 
However, here we need some sort of ``standard expression'' for the combination of 
angular momentum in the multi-shell compounds. Shall we choose (D|S) or (S|D)?
In our program, we require that for the same shell pair, the left side of shell
is always larger or greater than the right ones(see the operator < defined in
the shell class). We have a functions of ``sortAngPair'' to get the correct 
angular momentum combination for the multi-shell compounds.
 
\subsection{Shell Property Class}
 
Sometimes it could be quite common to get statistical data for a given 
shell property among all of shells in the molecule. For example, we want
to know that how many S type of shells in the molecule shell data. Therefore,
we set up this shell property class.
 
What's more, two shell properties could be paired together to form some more
advanced form. For example, in forming shell pairs we may need to know that
for a given angular momentum and the contraction degree how many shells on
it. Therefore, we designed the class of shell property pair to further 
describe that for a given shell property pair, the distribution status of 
of the given molecule shell. Here it's worthy to note, that in the given 
shell property pair, one shell property is worked as ``key'' and the other 
shell property is worked as ``value''. We note that there could be multiple 
value corresponding to one key.
 
However, the shell property class and shell property pair class is not sufficient
for practical use. For example, in forming the shell pair data we cares about
that for a given angular momentum pair and the contraction pair, that how many
shell pairs we can have? To answer such question we need to further consider 
the shell pairs distribution over the shell property pairs. So that's the origin
that we have this class of shell pair property.

Currently, we did not use these class in practical code. That is because we found 
a problem in real usage. For example, we are sorting shell pair according to the 
angular momentum and contraction degree. We can get distribution analysis through 
shell pair property class, but practically, since the shell pairs could be significant
or not significant, then it's actually contraction degree could vary. For example,
we have two S shells and each of them have 8 contracting Gaussian primitive functions.
However, in the final shell pair only 32 function pairs are significant, therefore
the contraction degree is not 64 but 32. Such information could not be anticipated 
without real calculation, and the shell pair property class fails on this point.

\end{comment}
 
