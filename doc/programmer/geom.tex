%
% set up on Feb. 2014
% 1 atom and element;
% 2 molecule class;
%   2.1  how to obtains its source data;
%   2.2  section number in molecule class;
%   2.3  indexing of atoms in molecule
%
\chapter{Geom Module}
%
%
%
This module handles the variety issues related to molecular geometry.
For example, the molecular geometry reading in/out, parsing the symmetry 
of a given molecular system; Cartesian and Z-Matrix transformation etc.

\section{Components List in Geom Module}

Geom module contains the following components:
\begin{description}
 \item [element] It defines the general property related to element 
 in periodic table.
 \item [Atom] This class defines a specific atom.
 \item [Molecule] This class defines a particular molecule through user's input file.
\end{description}

\section{Atom, Element and Molecule}
%
%
To define a geometrical structure in chemistry, the fundamental concept is 
atom and element. Element refers to the general properties for a given element
in periodic table, on the other hand; atom class defines a specific ``atom''
which locates in a given location in three dimensional space.

Molecule class is used to reflect the natural recognition of a ``MOLECULE''
in chemistry. In general, it's an aggregation of atoms.  Molecule object
is initialized based on user's providing data (e.g. input file).

Currently there's only one way to form an molecule object. That is, 
to read it through user's input data. It's intentionally designed to be in
such simple way so that to avoid other way of forming, for example; 
to form a molecule by adding atoms one by one. The additional way 
to form molecule may bring flexibility, however; it will also bring complexity
for forming properties based on molecule object (shell data, for instance).

However there's a special case which is referring to ``free atom''
molecule. In this molecule, it has only one atom. Such special molecule 
is useful in some cases (e.g. density matrix construction etc.).

\subsection{Section Number in Molecule Class}
%
%
In this program, each job is corresponding to one user's input file. However,
in the given input file there could be multiple geometrical structures defined.
These structures could be loosely connected, or closely interdependent. Therefore,
for each geometry an ``ID'' is provided, this ID is the section index for the 
given molecule in the input file. We note that for the molecule object, section
index begins from 1 and the following molecule section index is with an increment 
of 1. For the cluster object, since it's unique to the whole job (in the input file),
therefore it's section index is defined as 0.

Section number is the ``ID'' to distinguished the molecule objects in a running job.
Such ID could be derived into properties attached to the given molecule object, such
as shell and shell pair etc. With this unique ID (``uniqueness'' is only meaningful
for running job), it's able to discriminate different molecule objects.

\subsection{Index of Atom in Molecule}
%
%
In molecule, the atoms are aggregated in an order. Such order indicates that atom
retains a property of ``index'' inside the molecule. In molecule class, the
indexing is reflected by its position in atom vector.

Such index of atom provides a fundamental order to arrange data in program. For 
all of following properties attached to the molecule, e.g. the shell, shell pairs,
density matrix, molecular orbital etc. their arrangement of data are all based 
on the index of atom.



\begin{comment}

\section{Cluster}
%
%
In this program, cluster is used to express the ``super molecule''.  ``Super
molecule'' has two senses of meanings, first, it's the aggregation of
molecules; second, it defines how to link these molecules together.

Since cluster is a special case of molecule, therefore cluster class is a
derived class of molecule. The additional data in the cluster class is used to
express that ``How the cluster is formed by linking different molecular
pieces''. 


\end{comment}


