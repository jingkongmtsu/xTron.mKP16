%
% set up in 2014
% 1  BLAS and LAPACK;
% 2  special functions;
% 3  matrix class
%
\chapter{Math Library Module}
%
%
%
The math library module consist of four major components:
\begin{itemize}
 \item BLAS(Basic Linear Algebra Subprograms);
 \item LAPACK(Linear Algebra PACKage);
 \item matrix and its derived classes;
 \item special functions
\end{itemize}

\section{BLAS and LAPACK}
%
%
BLAS\footnote{see \url{http://en.wikipedia.org/wiki/Basic_Linear_Algebra_Subprograms}} 
and LAPACK\footnote{see \url{http://en.wikipedia.org/wiki/LAPACK}} are two widely 
applied mathematical libraries. They also form an important foundation for building the 
quantum chemistry software. The general idea for implementing the BLAS and LAPACK
functions in this program is to construct some general and flexible wrapper for 
calling the real working function from the vendors.

\subsection{BLAS and LAPACK Functions}
%
%
%
This program focus on the following functions provided by the BLAS and 
LAPACK:
\begin{description}
 \item [BLAS1] vector/vector and vector scaler operations;
 \item [BLAS2] matrix and vector multiplication;
 \item [BLAS3] matrix and matrix multiplication;
 \item [matrix factorization] LU decomposition for general matrix;
 and Cholesky decomposition for positively defined symmetrical matrix;
 \item [matrix inverse] performed on the base of LU decomposition and Cholesky 
 decomposition;
 \item [singular value decomposition for general matrix];
 \item [symmetric matrix eigenvalue and eigenvector problem];
 \item [solving linear equation]
\end{description}

\subsection{The Design of Wrapper Functions and Parallel Consideration}
%
%
%
The key in designing the wrapper functions for BLAS and LAPACK vendors
is flexibility. In general, the wrapper function is expected to adapt
to variety of vendor functions and wrapper function is able to fit
variety requirement for BLAS and LAPACK functions.

More specifically, the design of wrapper functions for BLAS and LAPACK
should considering the following factors:
\begin{description}
 \item [variety of vendors] The wrapper function should be able to fit
 different BLAS and LAPACK vendors;
 \item [parallel consideration] wrapper function should be able to 
 dynamically switch on/off multi-threading mode (see the following content
 for more discussion)
\end{description}

For adapting to variety of vendors, we use ``macro'' to control which 
BLAS/LAPCK vendor the program is trying to link against. This is because
in each compiling/linking time, only one BLAS/LAPACK vendor is accepted 
to be included in the program.

Parallelization is another important factor to consider. In this program
there are many places that needs BLAS and LAPACK functions. Some places
only deal with small size of matrices, for example; size varies from 20 
to 100. More importantly; the calculation could be carried out inside
threads, therefore such BLAS/LAPACK function are required to be performed
in sequential way. In the other places, the code is manipulating great 
size matrices (matrix dimension could be 10000, 20000 or even higher)
then it's beneficial to switch on the multi-threads mode and the computation
will be benefited from the use of multi-threading libraries. Therefore,
it's important that the program can get support from multi-threading BLAS/LAPACK 
libraries who are able to dynamically switch on/off multi-threads mode.

\subsection{Practical Implementation}
%
%
%
The general idea for implementing BLAS and LAPCK is simple: all of ``dirty''
work are done by the vendors, we just use them.

For BLAS1 operations, we implemented it by ourself by following the general
loop-unrolling skills \footnote{see \url{http://en.wikipedia.org/wiki/Loop_unwinding}
for more information} in the standard BLAS1 code. Such reduplication seems
conflicting our proposal of ``DRY(Do not repeat yourself)'' (see \ref{principle}
for more information), however; such choice is made after careful consideration.

In our program there are many places which require mixing the BLAS1 function
so to finish some complicated operation. For example, the following operation:
\begin{equation}
 w[i] += a*x[i]y[i]z[i] + w[i]
\end{equation}
need to invoke BLAS1 library several times. However, from programming view 
such operation is similar with the daxpy function since they are all element-wise 
operation. Therefore, by applying the similar code structure to this operation
it's able to create a new BLAS1 function that can perform the element-wise 
operation within only one loop. Because of the reason, we implement our own
BLAS1 functions and extend it as new operation requires.

The rest of BLAS functions (BLAS2 and BLAS3) and the whole LAPACK functions
are solely from vendors. We just construct the wrapper function and call
the vendor functions. In particular, if the vendor functions provide the 
muti-threading function which is realized with OpenMP, our wrapper function
is able to dynamically turn on/off the multi-threading mode by setting 
the number of threads \footnote{with the use of omp\_set\_num\_threads()
function. See the OpenMP reference.}.

%%%%%%%%%%%%%%%%%%%%%%%%%%%%%%%%%%%%%%%%%%%%%%%%%%%%%%%%%%%%%%%%%%%%%%%%%
\section{Implementing General Matrix Class}

\subsection{Order Arrangement in Multidimensional Matrix}
%
%
For the multidimensional matrix, in general there are two ways 
to store the data\footnote{see 
 \url{http://en.wikipedia.org/wiki/Row-major_order}}:
\begin{enumerate}
 \item row major;
 \item column major
\end{enumerate}

For an arbitrary multidimensional array, which is expressed as:
\begin{equation}
 M \Rightarrow  m[i][k][k][l]
\end{equation} 
row major means the last dimension $l$ is changing first, and column 
major means the first dimension $i$ is changing first.

In our program, the multidimensional array is actually stored in a
continuous one dimensional storage space, and column major is used 
for all the multidimensional arrays:
\begin{equation}
 A(d1,d2,d3,d4,...,dn) \Rightarrow d1+d2*S1+d3*S2+....+dn*Sn-1
\end{equation} 
$Sk$ is:
\begin{equation}
 Sk=D1*D2*D3....*Dk
\end{equation} 
$D_{i}$ is the $i$th dimension, and in C/C++ program the index starts from 0.

\subsection{Leading Dimension}
%
%
Suggest an array in dimension of $(5,6)$:
\begin{equation}
A = 
\begin{bmatrix}
1 &  6  &  11 &  16 &  21 &  26 \\ 
2 &  7  &  12 &  17 &  22 &  27 \\
3 &  8  &  13 &  18 &  23 &  28 \\
4 &  9  &  14 &  19 &  24 &  29 \\
5 &  10 &  15 &  20 &  25 &  30 \\
\end{bmatrix}
\end{equation} 

If B is a sub matrix of A:
\begin{equation}
B = 
\begin{bmatrix}
13 & 18 \\ 
14 & 19 \\ 
15 & 20 \\ 
\end{bmatrix}
\end{equation} 

Then for identifying B, the following information is needed:
\begin{enumerate}
	\item  B's original position: $A((3-1)+(3-1)*5)$
	\item  B's row(3) and column(2)
	\item  leading dimension is 5
\end{enumerate}

Therefore, leading dimension is used to identify the real 
position of the sub matrix in the whole matrix. For example,
suggest that B is the original position of $A((3-1)+(3-1)*5)$; 
the position in the whole matrix and the value in matrix B are 
corresponded as:
\begin{align}
13 &\Rightarrow  B[0+0]    \\ \nonumber  
14 &\Rightarrow  B[1+0]    \\ \nonumber 
15 &\Rightarrow  B[2+0]    \\ \nonumber 
18 &\Rightarrow  B[0+1*5]  \\ \nonumber 
19 &\Rightarrow  B[1+1*5]  \\ \nonumber 
20 &\Rightarrow  B[2+1*5]  
\end{align}
Thus to access to B's element the leading dimension is necessary.

\subsection{General Design for Matrix Class}
%
% 1 build the wrapper class;
% 2 call the lapack and blas to do real job
%
The implementation of matrix function could be divided into
two general categories:
\begin{itemize}
 \item functions to perform matrix operations;
 \item wrapper class which is to build matrix object, and manipulates
 the matrix operations
\end{itemize}

The matrix function in fact could be realized by a series of C style 
functions. The implementation of GNU Scientific Library\footnote{See 
\url{http://www.gnu.org/software/gsl/}} is a good example. The matrix
class exists only as a simple wrapper of these operations. In general,
the matrix wrapper class performs two important roles:
\begin{itemize}
 \item manipulates the memory hold by the matrix object;
 \item provides a clean and simple interface for the user of matrix class
\end{itemize}

Currently there are a lot of excellent libraries providing the matrix
functions (for example, Eigen, Armadillo etc.). In these libraries
they are handling the matrix in a general way (matrix element type is
generic), therefore generic programming skills is widely applied in these
libraries. The required matrix operation in this program, however; is only
a small subset of entire matrix operations. Furthermore, all of matrix
operations used in this program is concentrating on floating point numbers.
Therefore, we provide a self-own simple matrix class and in this class
it uses the STL vector to manipulate the memory. Here we note that the 
offering of this additional matrix class is only a purpose of convenience,
it's possible to construct similar matrix wrapper class on base of 
libraries of Eigen, Armadillo etc. as long as it satisfies the following 
requirement:
\begin{enumerate}
 \item all the matrix classes should have the same class name, and 
 public/protected method names; so that to keep the use of 
 matrix class consistent in the rest of the program;
 \item the matrix classes should support all of necessary matrix operations
 listed in \ref{matrix_operations};
 \item the matrix class should easily adding support for a new numerical 
 libraries working back-end; 
 \item the matrix class should support parallelization switch on/off per 
 user's request;
 \item the matrix class is able to be derived to form new matrix classes 
\end{enumerate}

\subsection{Functions for General Matrix}
%
%
\label{matrix_operations}
The functions for matrix used in this program could be divided into 
several categories:
\begin{description}
  \item [dimension change/memory manipulation]: 
  \begin{itemize}
   \item initialization/destruction of matrix;
	\item reset the dimension of matrix (enlarge matrix or shrink matrix);
	\item memory release
  \end{itemize}
  \item [fetching data]: 
  \begin{itemize}
   \item return dimension information;
   \item return constant floating point number pointer for a given row, column position;
   \item return non-constant floating point number pointer for a given row, column position;
   \item return matrix value for a given row, column position;
  \end{itemize}
 \item [data copying]:
 \begin{itemize}
  \item copy whole matrix to other matrix;
  \item replace the whole content of matrix from another matrix;
  \item copy part of matrix data to other matrix (copying in part between matrix);
  \item replace part of matrix data from other matrix (copying in part between matrix);
  \item copy matrix content to vector;
  \item load a vector's content to matrix
 \end{itemize}
 \item [level 1 BLAS operations]: adding, scaling, dot product etc.
 \item [matrix operation with itself]:
 \begin{itemize}
  \item matrix transpose;
  \item copy lower triangular part to upper triangular part (only for debug purpose);
  \item trace of matrix;
  \item matrix itself with its transpose added in;
 \end{itemize}
 \item [vector and matrix operations]: level 2 BLAS operations;
 \item [matrix and matrix operations]:
 \begin{itemize}
  \item symmetrical matrix multiplication;
  \item general matrix multiplication;
  \item submatrix multiplication;
  \item direct sum operation: $A = \bigoplus^{n}_{i=1}A_{i}$ 
 \end{itemize}
 \item [LAPACK functions]:
 \begin{itemize}
  \item eigenvalue and eigenvector solver for symmetrical matrix;
  \item power of matrix for symmetrical matrix;
  \item inverse of matrix for square matrix and symmetrical matrix;
  \item solve the rank of a matrix (based on singular value decomposition);
  \item linear equation solver: $AX = B$;
  \item orthogonal matrix forming (based on Lowdin orthogonalization etc.)
 \end{itemize}
 \item [other matrix utilities]:
  \item matrix printing;
  \item matrix difference compare (only for debugging purpose);
  \item writing/reading matrix from files
\end{description}

Here it's worthy to note, that in the matrix power function and orthogonal 
matrix forming procedure, the input matrix could be in a state of ``linear
dependent'' therefore the matrix is not a full rank matrix. Similar cases 
is also existing for general matrix inverse based on LU decomposition. 
In these cases the linear dependent vectors are trying to be removed, and 
such technique is similar with the derivation of ``Moore–Penrose pseudoinverse''
\footnote{ see the page of \url{http://en.wikipedia.org/
wiki/Moore\%E2\%80\%93Penrose_pseudoinverse} for more details}. The user should 
be particular careful about this case, since the result matrix could be diverged 
quite a lot from the one user expect to get. Please check the result if linear
dependency exists in the matrix!

\subsection{Matrix in Threads}
%
%
%
\label{matrix_threads}

In the integral calculation\footnote{see the section \ref{parallel_imp_integral}
for more information}, the matrix is extensively used within threads. Typically,
the use of matrix in threads is divided into two kinds:
\begin{enumerate}
 \item matrix is used as local object inside the threads (local threading matrix);
 \item matrix is used as output between threads
\end{enumerate}

The matrix used locally in threads is typically a normal matrix except that
its memory management should be thread safe and scalable to the number of 
threads. For achieving the scalability of memory management between threads,
the matrix class uses the scalable memory allocator\footnote{see the discussion
in \ref{mem_management}}.

Since the scalable allocator is also applied to single threads environment,
the matrix class uses the scalable allocator for both single-thread and 
multi-threads purposes.

When matrix is used as output between threads, there could be possible performance 
lost due to the ``false sharing'' in the caches when different threads are 
referring to data (this happens even if the threads refer to different region
of data) in the matrix. This is because each thread has it's own copy of the caches
and system requires the caches to be updated among all of threads. In this case,
decreasing the granularity size in parallel will provide some help. 



