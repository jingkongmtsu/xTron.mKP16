
%
% set up July 2013
%
\chapter{Van der Waals Interaction Calculation}


\section{XDM Model}
%
%


\subsection{Hirshfeld Weights}
%
%
%
Similar with the partition weights in DFT calculation, Hirshfeld weights 
is used to partition the molecular property into atom based weights. 
For achieving such purpose, Hirshfeld weights employs the free atom
density:
\begin{equation}
\rho^{atom}(r) = \sum_{\mu\nu} P_{\mu\nu}^{free} \phi_{\mu}(r)\phi_{\nu}(r)
\end{equation} 
Here $\phi$ is the batch basis set, and $P_{\mu\nu}^{free}$ is the free
atom density matrix generated from free atom SCF calculation. 
The free atom density uses the both ``local'' and ``non-local'' data
(local means the data is only related to the atom itself, and non-local
implies the data is related to the whole molecule environment). Therefore,
there will be a data matching work between the two types of data.

Based on the free atom density, Hirshfeld weights could be expressed as:
\begin{equation}
  W_{i}(r) = \frac{\rho^{atom}(r)}{\sum_{n}\rho^{atom}_{n}(r)} 
\end{equation} 



